% =============================================================================
% FILE NAME : 01_fundamentals.tex
% DEPARTMENT: University of Tuebingen
% AUTOR     : Paul Palomero Bernardo & Konstantin Lübeck
% =============================================================================
% CONTENT   : Include for chapter "Fundamentals"
% =============================================================================

Ziel dieses Kapitels ist eine Einführung in die Thematik BlaBlaBla ...

\section{RISC-V}

RISC-V \cite{riscv} is an open standard Instruction Set Architecture (ISA).
RISC-V International is the non profit that provides the open, royalty-free ISA that is RISC-V.
\\\\
RISC-V was originally developed to support computer architecture research and education, but the authors now
aim for RISC-V to also be used in industry implementations \cite{riscv_spec}.

\subsection{Unterabschnittsüberschrift}

BlaBlaBla ...


Bevor wir uns der Auswertung bzw. Bewertung der gewonnenen Primärdaten zuwenden, wollen wir zunächst einige grundlegende Begriffe der deskriptiven Statistik wiederholen.
\section{Stichproben}

Grundsätzlich haben wir es bei Microarrayexpressionsdaten mit einer {\textit Stichprobe} aus einer {\textit Population (Grundgesamtheit)} zu tun.   
Wir bezeichnen nun im allgemeinen mit $X=\{x_1,x_2,\ldots,x_n\}$ die Beobachtungsdaten vom Umfang $n$. 
Diese Daten sollen mit statistischen Kenngrößen beschrieben werden. Aus diesen will man möglichst zuverlässig auf die zugrundeliegende Verteilung in der Grundgesamtheit schließen. Hierzu verwenden wir die {\textbf Lage-} und {\textbf Streuparameter}. Zunächst wenden wir uns aber der Häufigkeits- und Summenhäufigkeitsverteilung zu, die sowohl graphisch als auch numerisch einen Eindruck über die Verteilung von $X$ bieten. Dafür betrachten wir diskrete Verteilungen.

Gegeben sei eine Stichprobe $(X_1,X_2,\ldots,X_n)$. Eine Funktion $Z_n=Z(X_1,\ldots,X_n)$ wird als {\textit Stichprobenfunktion} bezeichnet. Sie ist selber eine Zufallsgröße.

\subsection{Häufigkeiten und Histogramm}
In der Stichprobe $X$ trete der Wert $x_i$ genau $n_i$ mal auf, $i=1,2,\ldots m$. Dann ist $\sum_i n_i = n$. Der Quotient $n_i/n$ ist die {\textit relative Häufigkeit} für das Eintreten des Ereignisses ``$X=x_i$''.
Die Menge der relativen Häufigkeiten $\{n_1/n,n_2/n,\ldots, n_m/n\}$ heißt {\textit Häufigkeitsverteilung} von $X$. Ferner heißt die Menge $\{s_1,\ldots,s_m\}$ mit $s_i=\sum_{k=1}^{i}n_k/n$ die {\textit Summenhäufigkeitsverteilung} von $X$.

Für die graphische Darstellung der Häufigkeitsverteilung wird das {\textit Histogramm} gewählt. für die Summenhäufigkeitsverteilung die {\textit Treppenfunktion}.

Wenn wir natürlich Zahlen mit Komma haben, so sieht das in Deutsch irgendwie seltsam aus, z.B. $0,7$. Ich nehme dafür den SIstyle: $\SI{0,7}{}$ usf.  Oder nett ist auch: \SI{9.3e5}{km}

\todo{Mit diesem Kommando kann man eigene Kommentare am Spaltenrand hinzufügen.}
\note{Mit diesem Kommando kann man eigene Kommentare in der Textspalte hinzufügen.}

%
\subsection{Wichtige Verteilungen}

\subsubsection{Die Normalverteilung}
Die Dichte der Normalverteilung ist gegeben durch
\begin{equation}\label{eqn:dichtenormal}
g(x) = \frac{1}{2\pi\sigma}\cdot e^{-\frac{(x-\mu)^2}{2\sigma^2}}
\end{equation}
wobei $\mu$ (Lage) der Mittelwert und $\sigma$ (Breite) die Standardabweichung der Normalverteilung ist. 
Durch die $z$-Transformation lässt sich die Normalverteilung auf die Standardnormalverteilung mit $\mu=0$ und $\sigma=1$ transformieren.

Die Normalverteilung bildet die Basis fast der gesamten statistischen Theorie. \footnote{ 
	\glqq Everyone believes in the normal law, the experimenters because they imagine it is a mathematical theorem, and the mathematicians because they think it is an experimental fact.\grqq{} (Gabriel Lippman, in Poincar's Calcul de probabilité, 1896)}. Auch bei der Analyse der Microarraydaten werden wir sehr oft von der Annahme der Normalverteilung Gebrauch machen. Allerdings sollten wir uns klarmachen, dass  rein experimentell zahlreiche Untersuchungen gezeigt haben, dass die echten Fehler selten, wenn überhaupt normal verteilt sind.


\section{Schätzung von Parametern}
Allgemein erhofft man sich beim Ziehen einer Stichprobe, einen unbekannten Parameter $\gamma$ der Grundgesamtheit, z.B. den Mittelwert, aus der Stichprobe zu schätzen.
\subsection{Eigenschaften von Punktschätzungen}

