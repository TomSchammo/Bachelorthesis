% =============================================================================
% FILE NAME : 00_introduction.tex
% DEPARTMENT: University of Tuebingen
% AUTOR     : Tom Schammo
% =============================================================================
% CONTENT   : Include for chapter "Introduction"
% =============================================================================


Rust \cite{rustlang} is a somewhat new language, with their very first release in early 2012.
But the 1.0 alpha release was only in early 2015 with the full 1.0 release following a few
months later in Mai of the same year \cite{rust_releases}.
C, as a comparison has been used as early as the 1970s.\\
But Rust has been gaining in popularity \cite{rust_popularity} over the last few years, however
due to its relatively 'young' age there still are huge gaps when it comes to software support and available libraries.
Embedded systems is an area where Rust has the potential to be very useful due to it being very performant as well as memory safe,
thereby being a potential alternative to C or C++, especially in high stakes, high performance, real time applications.
However, that gap is very noticeable when it comes to embedded devices.\\
Additionally, with the improvement of machine learning technologies and their incorporation into IoT and smart devices,
the use cases for hardware accelerated devices are ever-growing.
As voice controlled devices, specifically voice assistants (VAs) have become almost omnipresent,
with Siri being shipped with every Apple smartphone, tablet and laptop, Google Assistant being installed
on every Android phone (except for a few custom ROMs that remove Google software) and Cortana being integrated
into every new Windows operating system, users become increasingly used to these technologies.
The adoptions of voice controlled devices can also be observed in private homes with standalone
devices like the Amazon Echo to control appliances and a variety of IoT gadgets.
However, in the IoT space, security is more often than not, not even an afterthought.
This is mostly a result of software bugs that could be avoided by enforcing memory safety.
So in my thesis I'll improve upon the Rust ecosystem by expanding upon the thesis of Raphael Vogelgsang \cite{rust_pulp}
and implementing support for the UltraTrail \cite{ultratrail} AI accelerator.
\newpage
This thesis is structured as follows:
The Chapter~\ref{cha:fundamentals} will provide a fundamental overview of RISC-V, 'Embedded AI', (embedded) Rust and microphone technology.\\
Chapter~\ref{cha:related_work} covers previous work on the PULPissimo, and goes a bit into the UltraTrail architecture.
Finally, it covers the state of keyword spotting in the industry.
It briefly goes into its use cases and then covers one of them a little more in depth.
The concept of the thesis is covered in Chapter~\ref{cha:concept}.
That chapter briefly introduces what I'm trying to do in my thesis.
After that it covers the microphones used and finally goes into the tests that I have used
to assess the functionality of the UltraTrail and microphone drivers.
Chapter~\ref{cha:results_and_discussion} begins with a discussion of the implementation
of the UltraTrail driver as well as the PDM and $I^2S$ microphones.
This is followed by a presentation of the driver functionality tests and a brief discussion of the results.
Finally, Chapter~\ref{cha:conclusion_and_future_work} concludes this thesis and discusses any shortcomings,
as well as what could be built upon what has been accomplished.
