% =============================================================================
% FILE NAME : results_and_discussion.tex
% DEPARTMENT: University of Tuebingen
% AUTOR     : Tom Schammo
% =============================================================================
% CONTENT   : Include for chapter "Results and Discussion"
% =============================================================================


\section{Microphones}

\subsection{Implementation}

\subsection{Experiments}


\section{UltraTrail}

\subsection{Implementation}

To implement the driver for UltraTrail, it first has to be added to the SVD file in the PAC
as a peripheral to generate the necessary Rust code to access hardware registers.
Subsequently, the driver can be implemented in the HAL crate.
Finally, I have ported a small program from C to Rust, that contains a set of pre-defined
weights, bias and feature inputs, as well as expected results.
This program can be used to test the implementation of the driver, as it also
performs different configurations by reading and writing the different registers.
\\\\
However, there have been a few differences when it comes to the implementation details.
Most importantly, the C drivers for the PULPissimo provide the \lstinline{__rt_periph_wait_event} function,
which is basically a sophisticated busy wait.
The test code uses that function to wait for UltraTrail to trigger an event interrupt, to announce that
it is done processing as shown in Listing~\ref{code:c_wait_for_event}.

\begin{lstlisting}[style=colorEX,language=C,caption={Waiting for the event in C},label={code:c_wait_for_event}]
pacc_start();

soc_eu_fcEventMask_setEvent(ARCHI_SOC_EVENT_FCHWPE0);
__rt_periph_wait_event(ARCHI_SOC_EVENT_FCHWPE0, 1);

pacc_stop();
\end{lstlisting}

However, that function has not been implemented in the Rust drivers, so I had to slightly
vary the implementation of the test code.\\
First I had to define a Mutex \footnote{\url{https://doc.rust-lang.org/std/sync/struct.Mutex.html}}
for the event state, to monitor whether an event interrupt has been triggered as demonstrated
in Listing~\ref{code:mutex}.

\begin{lstlisting}[style=colorEX,language=Rust,caption={Definition of the necessary Mutexes},label={code:mutex}]
static EVENT_STATE: Mutex<Cell<bool>> = Mutex::new(Cell::new(false));
\end{lstlisting}

Subsequently, I defined an event handler as presented in Listing~\ref{code:event_handler}, which
stops the accelerator when it is done and changes the global event state Mutex.

\begin{lstlisting}[style=colorEX,language=Rust,caption={Event handler function},label={code:event_handler}]
fn hwpe_event_handler() {
    interrupt::free(|cs| {
        unsafe {
            Accelerator::stop_steal();
        }

        EVENT_STATE.borrow(*cs).set(true);
    })
}
\end{lstlisting}

After that I just had to enable the event and set the event handler to deal with the event when it is triggered.
The code of that can be seen in Listing~\ref{code:enable_event}.

\begin{lstlisting}[style=colorEX,language=Rust,caption={Code to enable the hardware event and set the handler},label={code:enable_event}]
unsafe {
    event_interrupt.enable_event(EventId::Hwpe0);
    event_interrupt.set_event_handler(EventId::Hwpe0, hwpe_event_handler);
}
\end{lstlisting}

Finally, I could write the busy wait, which basically does nothing until the interrupt is received
and then changes the event state.
An example of that is displayed in Listing~\ref{code:busy_wait}

\begin{lstlisting}[style=colorEX,language=Rust,caption={Snippet of the busy wait that waits for UltraTrail to finish},label={code:busy_wait}]
accelerator.start();

let mut event = false;
while !event {
    wait_x_nops(1000);

    interrupt::free(|cs| {
        if EVENT_STATE.borrow(*cs).get() {
            event = true;
            EVENT_STATE.borrow(*cs).set(false);
        }
    })
}

// ...
\end{lstlisting}

\subsection{Experiments}

The expected results are stored in a vector containing 12 16bit integers, displayed in the Listing~\ref{code:acc_results}.

\begin{lstlisting}[style=colorEX,language=Rust,caption={The expected results from the driver test},label={code:acc_results}]
let correct_results = [
    0xf246, 0x16bf, 0x08a0, 0xfe52, 0xf709, 0xef58, 0xff6b, 0xf946, 0xfebd, 0x0f40, 0xee45,
    0x18f0,
];
\end{lstlisting}

When comparing the code in Listing~\ref{code:acc_results} to the minicom output shown in \ref{code:minicom_out_acc},
one realizes that they match. Which leads me to believe that the UltraTrail driver works as intended.

\begin{lstlisting}[style=colorEx,caption={Minicom output after executing the driver test},label={code:minicom_out_acc}]
result #0: 0xf246
result #1: 0x16bf
result #2: 0x8a0
result #3: 0xfe52
result #4: 0xf709
result #5: 0xef58
result #6: 0xff6b
result #7: 0xf946
result #8: 0xfebd
result #9: 0xf40
result #10: 0xee45
result #11: 0x18f0
\end{lstlisting}


% Hier sollen die erreichten Ergebnisse vorgestellt werden.
% Hierzu zählt die Vorstellung des Versuchsaufbaus sowie die geeignete Aufbereitung und Diskussion der Ergebnisse.
% Mehrwert oder Nutzen benennen!
